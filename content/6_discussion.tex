\chapter{Discussion} \label{sec:discussion}

For this, we will discuss each contribution, and draw a conclusion in the end.

\section{\modelname{}}

With \modelname{}, we propose a high-level time-to-power model for workloads.
In essence, it models a workload as having startup phases and a work phases. 
Each of these then have a constant power attached to it. 
This presents a superset of the workload models used in prior literature in this field.
Using a constant per phase is a big simplification, however. 
The main motivation for this was to reduce complexity in the LP suspend \& resume scheduler.
We also argue that in the context of the carbon-emissions resolution being low, this is appropriate.

A drawback of \modelname{} is that power used outside the job is not captured at all.
In a real-world cloud scenario, there will be e.g. be VM startup times \cite{zheng_benchmarking_2019}.
Also, we assume workloads to be executed in an isolated matter. 
Thus, each workload under \modelname{} increases power and carbon emissions linearly.
On real-world hardware however, servers have a baseline energy demand. 
Sharing resources and increasing utilization of a server increases energy efficiency \cite{barroso_case_2007}.

In the context of long workloads and parallel execution, the clean-cut phases we observed on a single machine and a short job may 
not hold in HPC environments.

\section{\programname{}}

Through an iteration on an existing testbed, GAIA, our implementation of \programname{} enables simulating workloads in the context of minimal carbon-emissions.
In comparison to prior work, workloads now have heterogeneous phases and their startup times are part of the scheduling.
We have tried to evaluate this changed assumption for different startup costs, different phases, waiting times, and job lengths. 
The trend that allowing resumeable jobs to be deferred for longer increases carbon savings \cite{wiesner_lets_2021,hanafy_going_2024, hanafy_war_2023}, appeared in our results as well. 
We were also able to observe the findings by Sukprasert et al. \cite{sukprasert_limitations_2024} that a suspend \& resume strategy benefits very long jobs more.

Wholly unexplored are the effects job heterogeneity that we added.
The reason being that our chosen scenarios:

\begin{itemize}
    \item Had phases of relatively short length. It is likely that longer phases result in more pronounced results.
    \item Used too many phases. We chose to repeat low and high phases until the given job length is met, meaning that the amount of phases increased linearly. As suspend \& resume scheduling works better for very long jobs, the expected effects could not be computed within a 20 minute scheduling time.
\end{itemize}

A drawback of the LP implementation is that computation time is dependent on the shortest phase in \modelname{} (see Section \ref{sec:checkpoint_resume_lp}). Thus, if startup times are short, they may need to be removed in favor of runtimes. The OPR approach described in Section \ref{sec:state_of_the_art}, which assumes startup to have a cost but no length, does not have this problem.

We want to propose a more fit approach to the evaluation:
First, a literature review of long-running workloads with long phases is necessary. Additionally, the power measurement capabilities of cluster nodes we described in Section \ref{sec:power_measurements} could be used to measure larger ML models than we were able to execute on our private hardware.
Given these workloads, the evaluation is then done under different carbon traces to remove bias.
In our evaluation, the days 4 and following had an influx of wind power which dominated the carbon savings for the longest waiting time. 

In our case, due to the way each job was generated and indexed, the hardest problems were executed last. 
A better approach would be to compute the complex problems first to check whether time limits are hit.
Running a sample evaluation for select parameters should also have been done.

\section{Future Work} \label{sec:future_work}


\todo[inline]{Discuss some stuff!}