% => Wenn die Arbeit auf Deutsch verfasst wurde, verlangt das Studienreferat KEINEN englischen Abstract

 % englischer Abstract
\null\vfil
\begin{otherlanguage}{english}
\begin{center}\textsf{\textbf{\abstractname}}\end{center}

\noindent Hallo!

\todo[inline]{Write Abstract}
\todo[inline]{Fix Gutachter on titlepage}

\end{otherlanguage}
\vfil\null


% => Wenn die Arbeit auf Englisch verfasst wurde, verlangt das Studienreferat einen englischen UND deutschen Abstract (der dt. Abstract kann dann ggf. auch ans Ende der Arbeit)

% deutsche Zusammenfassung
\null\vfil
\begin{otherlanguage}{ngerman}
\begin{center}\textsf{\textbf{\abstractname}}\end{center}

\noindent 

Lorem ipsum dolor sit amet, consectetur adipiscing elit, sed do eiusmod tempor incididunt ut labore et dolore magna aliqua. Ut enim ad minim veniam, quis nostrud exercitation ullamco laboris nisi ut aliquip ex ea commodo consequat. Duis aute irure dolor in reprehenderit in voluptate velit esse cillum dolore eu fugiat nulla pariatur. Excepteur sint occaecat cupidatat non proident, sunt in culpa qui officia deserunt mollit anim id est laborum.

%- temporal shifting
%- include the costs of pausing a job (saving work and resuming it)
%- include costs of shutting down or starting additional compute nodes
%- include frequency scaling 
%- provide a model for simulating scheduling of jobs, and also model what information need to provide to increase CO2 efficiency. 

%C The study aims to [state the objectives or goals of the research]. Through [methodology/approach], %data was collected from [source or population]. The findings reveal [main results or discoveries].

%The analysis highlights [key insights or implications], shedding light on [broader significance or relevance]. This research contributes to the understanding of [specific field or subject area], offering [potential applications or recommendations].

%Overall, this thesis provides valuable insights into [topic] and offers avenues for further exploration in [related areas].

\end{otherlanguage}
\vfil\null



