\chapter{Discussion} \label{sec:discussion}

For this, we will discuss each contribution, and draw a conclusion in the end.

\section{\modelname{}}

With \modelname{}, we propose a high-level time-to-power model for workloads.
In essence, it models a workload as having startup phases and work phases. 
Each of these then have a constant power attached to it. 
This presents a superset of the workload models used in prior literature in this field.
Using a constant per phase is a big simplification, however. 
The main motivation for this was to reduce complexity in the LP suspend \& resume scheduler.
We also argue that in the context of the low carbon-emission resolutions, this is appropriate.

A drawback of \modelname{} is that power used outside the job is not captured at all.
In a real-world cloud scenario, there will be e.g. be VM startup times \cite{zheng_benchmarking_2019}.
Also, we assume workloads to be executed in an isolated matter. 
Thus, each workload under \modelname{} increases power and carbon emissions linearly.
On real-world hardware however, servers have a baseline energy demand. 
Sharing resources and increasing utilization of a server increases energy efficiency \cite{barroso_case_2007}. 
Performance impacts due to shared hardware is also not part of \modelname{}.

In the context of long workloads and parallel execution, the clean-cut phases we observed on a single machine and a short job may 
not hold in HPC environments.

\section{\programname{}}

We iterated on an existing testbed, GAIA.
In comparison to prior work, workloads in \programname{} now have heterogeneous phases and their startup times are part of the scheduling.
We have tried to evaluate this changed assumption for different startup costs, different phases, waiting times, and job lengths. 
The trend that allowing resumeable jobs to be deferred for longer increases carbon savings \cite{wiesner_lets_2021,hanafy_going_2024, hanafy_war_2023}, appeared in our results as well. 
We were also able to observe the findings by Sukprasert et al. \cite{sukprasert_limitations_2024} that a suspend \& resume strategy benefits very long jobs more.

Wholly unexplored are the effects job heterogeneity that we added.
The reason being that our chosen scenarios:

\begin{itemize}
    \item Had phases of relatively short length. It is likely that longer phases result in more pronounced results.
    \item Used too many phases. We chose to repeat low and high phases until the given job length is met, meaning that the amount of phases increased linearly. As suspend \& resume scheduling works better for very long jobs, they could not be scheduled within a time limit of 20 minutes.
\end{itemize}

A drawback of the LP implementation is that computation time is dependent on the shortest phase in \modelname{} (see Section \ref{sec:checkpoint_resume_lp}). Thus, if startup times are short, they may need to be removed in favor of runtimes. The OPR approach described in Section \ref{sec:state_of_the_art}, which assumes startup to have a cost but no length, does not have this problem.

We want to propose a more fit approach to the evaluation:
First, a literature review of long-running workloads with long phases is necessary. Additionally, the power measurement capabilities of cluster nodes we described in Section \ref{sec:power_measurements} could be used to measure larger ML models than we were able to execute on our private hardware.
Given these workloads, the evaluation is then done under different carbon traces to remove bias.
In our evaluation, the days 4 and following had an influx of wind power which dominated the carbon savings for the longest waiting time. 

In our case, due to the way each job was generated and indexed, the hardest problems were executed last. 
A better approach would be to compute the complex problems first to check whether time limits are hit, or to run them in a random order.
Running a sample evaluation for select parameters should also have been done.
As of now, the impact of our used parameters on the resulting LP complexity is not entirely clear.
A runtime analysis on \programname{}' LP scheduler could be a guide on what can be scheduled.

\section{Conclusion}



\section{Future Work} \label{sec:future_work}

There are several avenues for future work.
Regarding \programname{}, our suspend \& resume implementation is not yet able to make use of \modelname{}'s checkpoint annotations. Future work includes evaluating carbon-aware scheduling when jobs can only be checkpoint at some points.

Our LP scheduler showed long runtimes for more complex scheduling problems.
Future work could deal with either improving model performance \webcite{web_gurobi_threads,web_gurobi_optimization} or evaluating whether a break-even point in scheduling-time and carbon-saving exists.

We found that suspend \& resume scheduling works best for long jobs.
Future work could investigate which real-world workloads exhibit power heterogeneity.
Their saving potential our workload model could then be analyzed.

Instead of only testing predefined workloads, our LP scheduler could be combined with current research in the field of workload analysis. 
Köhler et al. \cite{kohler_recognizing_2021} classify workloads based on power signatures. 
On live systems, such an approach could be used to opt-in to a suspend \& resume strategy. 
We found that short jobs exhibit little potential carbon savings. 
If a workload then runs longer, classifying them as suspendible could lead to hybrid scheduling strategies.  
