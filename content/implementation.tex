\chapter{Implementing Schedulers using the New Model}

Now that an improvement on the job model has been made, the question, on how to evaluate the implications of said model, remains. 

% We first need to explain why we chose our approach (building upon exisiting work inside GAIA). The other option that is not using a simulation would be to schedule real jobs, for example by creating a Slurm plugin.

% We can then evaluate how well a Slurm plugin would work for our given Forschungsragen. End that section by deeming the plugin idea as unfit, we can then shift to arguing for the simulation approach as that is also something that just came out in related work (perhaps we should see wether we list GAIA as related work or introduce it just then)

\section{Choosing an Implementation Approach}

\subsection{Carbon-aware scheduling via a Slurm plugin}
\label{subsec:slurm_plugin}

My first idea was to use a non-simulation approach. 
The HPI's Data Lab \footnote{\url{https://hpi.de/forschung/infrastruktur/hpi-data-engineering-lab.html}} runs a \emph{Slurm} cluster and also has some nodes with power measurement infrastructure included. 
Slurm is an "open source, fault-tolerant, and highly scalable cluster management and job scheduling system for large and small Linux clusters"\footnote{\url{https://Slurm.schedmd.com/overview.html}}. 
The job scheduling part is important here, it also supports a plugin infrastructure that includes scheduling plugins. 
One of the highlighted papers\cite{inigo_goiri_greenslot_2011} in the related work section specifically used Slurm for its carbon-aware scheduler implementation and thus seemed like a good starting point for my own work.

\paragraph{Installing Slurm locally}

For my purposes, a local installation would suffice as I would not need to run heavy workloads but instead just the scheduling part of Slurm. 
While there is a Slurm \verb|apt get| package for my Ubuntu version, this would not work as any plugins to be included during the Slurm compilation, meaning I would have to do the same.

Slurm's documentation provides some guidelines on how to install Slurm, which I followed. 
I tried cloning the main branch, but I got stuck during the compilation process. However, using the predefined released versions worked.

Installing \verb|munge|\footnote{\url{https://github.com/dun/munge/wiki/Installation-Guide}} is also necessary, which is used for authentication in Slurm.

One problem arose as I tried to start the \verb|Slurmd|- and \verb|Slurmctld| services. The first one is the worker service that would later execute jobs submitted to Slurm. The latter is the main controller that, for example, schedules jobs on workers. While the command to start them would not fail, upon node inspection via Slurm's \verb|scontrol| command, it would show that all nodes are \verb|DOWN| instantly.
|
Dealing with Slurm's problems usually leads to inspecting its logs, in my case the logs showed the following:

\begin{lstlisting}[language=bash, frame=single, numbers=none, caption={Slurm error logs}, basicstyle=\ttfamily]
$ less config.log

error: Couldn't find the specified plugin name for cgroup/v2
    looking at all files
error: cannot find cgroup plugin for cgroup/v2
error: cannot create cgroup context for cgroup/v2
error: Unable to initialize cgroup plugin
error: Slurmd initialization failed

\end{lstlisting}

Slurm uses Linux' \verb|cgroup| feature to manage the submitted job's hardware ressources. The log hints at some problem related to Slurm's usage of it.
The solution was this was to provide Slurm's \verb|cgroup.conf| file. In my use-case of getting Slurm to simply start, this did not need to be very sophisticated so I just used an off-the-shelf (off-the-stackoverflow\footnote{\url{https://stackoverflow.com/a/74211989}}) configuration file.

Running \verb|scontrol| again, I was finally able to see idling nodes, meaning that Slurm was successfully installed form source:

\begin{lstlisting}[language=bash, frame=single, numbers=none, caption={Slurm running}, basicstyle=\ttfamily]
$ scontrol
scontrol: update NodeName=vincent-Laptop STATE=RESUME
\end{lstlisting}

\paragraph{Creating a scheduler plugin}

The Slurm documentation provides a short guide on how to add a plugin to Slurm.\footnote{\url{https://Slurm.schedmd.com/add.html}}. 
As a start, I simply copied Slurm's default scheduler (which is also a plugin) to to the specified directory under a new name, and addings that new name to Slurms build files. 
It was then was time to recompile Slurm. 
Now however, during the recommended \verb|autoreconf| step, an error occured:

\begin{lstlisting}[language=bash, frame=single, numbers=none, caption={Plugin recompilation errors}, basicstyle=\ttfamily]
$ autoreconf
auxdir/x_ac_sview.m4:35: warning: macro 'AM_PATH_GLIB_2_0' 
    not found in library
configure:25140: error: possibly undefined macro: AM_PATH_GLIB_2_0
      If this token and others are legitimate, 
      please use m4_pattern_allow.
      See the Autoconf documentation.
autoreconf: error: /usr/bin/autoconf failed with exit status: 1
\end{lstlisting}

The solution, while not very obvious, was to install the \verb|libgtk2.0-dev| library. \footnote{\url{https://stackoverflow.com/questions/7805815/autoconf-error-on-ubuntu-11-04}}. 
I then added a simple logging which then showed up in Slurm's log files too.

\paragraph{Adding more logic to the scheduler plugin}

One very helpful step for developing inside Slurm is to enable the debugging flags.
This must be decided before compilation by using the \verb|--enable-developer| and \verb|--disable-optimizations| flags during the \verb|/configure| step. 
With that, debug symbols are added to the outgoing binaries. 
As I was using vscode, I could then attach its debugger to the running Slurm thread with full functionality.

The code of the plugin runs in its own thread and there is no sandboxing or similar around it.
Thus there are seemingly no limitation on what can be done inside the plugin. 
For testing, I read out information on the incoming jobs such as set constraints or the user supplied comments. Terminating the jobs was also possible inside the plugin.

\paragraph{Problems of a scheduler plugin}

One big problem manifested in that not all jobs "showed up" inside the plugin's job queue. 
If I would submit 6 jobs, via Slurm's \verb|squeue| command, only a part such as the last 3 could be logged inside the plugin.
A possible explaination for this could be Slurms scheduler architecture: while there is a scheduler plugin, there also is a scheduler inside Slurm's main loop. 
That main scheduler loop also uses the same job queue as the plugin.

To hack around this, I tried disabling Slurm's main scheduling loop by setting \verb|sched_interval=-1| inside the Slurm configuration file. 
While this had the effect of being able to access all incoming jobs inside the plugin, it also had the side-effect of disabling all logic concerning starting the jobs.
So by choosing this route, the plugin would apparently also need to re-implement alot of extra logic, which conventionally would be not be put inside the scheduler plugin. 

I also looked into whether there were any API hooks that are exposed to the plugin. 
Up until Slurm version 20.11, scheduler plugins had callbacks such as when jobs were submitted. There also was support for "passive" scheduler that would get invoked when determined by Slurm.
The version I tested however,  23.11, removed all such functionality and documentation. All plugins are implemented via threads that only have callbacks on when they are started and stopped.

Thus, since there was no apparent way of getting around this scheduler race condition between the plugin and the main loop. 
The scheduler approach as a whole was dropped. 
While I would not say that a plugin approach is impossible, the effort to implement one from scratch subjectivly seems very high. 
The public documentation for developing on Slurm is scarce. 
There is a mailing list that can be searched, but it looks to be mostly aimed for administrating Slurm and not developing it.

Other avenues that could be explored are Slurm's Lua plugins. There is also a \emph{Slurm simulator}\footnote{\url{https://hpckp.org/articles/how-to-use-the-Slurm-simulator-as-a-development-and-testing-environment/}} which could potentially be used for carbon-aware scheduling simulation, but that I did not look into much for reasons of little documentation and seeming lack of continued support.

\subsection{Using a Simulation approach}

In a recently released paper by Hanafy et. al \cite{hanafy_going_2024}, a prototype testbed for simulating job scheduling on cloud providers is implemented. 
These Jobs could be executed on spot instances (cheap VMs that seek to increase cloud utilization), on-demand instances (short-notices VMs that are thus more expensive) or pre bought VMs (medium cost, but may be wasted), the paper then discussed balancing carbon- and dollar costs. 
They included a small notion notion of hardwre requirements in the form of required CPUs per job, but that was only used to scale the cost; all hardware requirements were abstracted away in the form of the cloud always having computing ressources available.

The important part is that they also included an implementation of some schedulers used in the highlighted related works. Among that an implementation for WaitAWhile\cite{wiesner_lets_2021}.
This meant that I could add the improved job model to that exisiting testbed and have something to compare against aswell.

\paragraph{Description of the GAIA simulator}

I would first like to describe the existing testbed in detail to make it clear which part is my work and which is not. A class diagram is provided in \ref{fig:class_diagram} which is heavily inspired by the existing architecture diagram in the original paper. 

\begin{figure}
    \includegraphics[width=\linewidth]{images/MA Thesis Diagram.pdf}
    \caption{Class Diagram of GAIA, red indicates removal or reduction, green indicates additions}
    \label{fig:class_diagram}
\end{figure}

The main part of GAIA is the scheduler. At program start, it takes parameters\todo{I should probably give an overview of which parameters are supported, so I have that terminology later} that determine the scheduling, such as whether tasks can be interrupted, how to use the carbon information (e.g. use a perfect-information / oracle approach or to use a running average), and how to balance the dollar cost for scheduling. 
The latter I removed for simplicity, which is marked in the figure via the red color of the reserved and spot instances. In my case of carbon-aware scheduling, I did not need that feature.

In the original implementation, the scheduler had multiple queues of tasks. 
The paper presented an approach of a short queue (tasks under 2 hours) and long tasks (all other), short tasks would get scheduled on spot instances. 
Again, as I already removed the spot instance, feature, I did not need the queue multiplicity and it was removed.

The tasks queues are based on historical traces. GAIA provided multiple traces that are described in the paper. To those I added a trace of the HPI's Data Lab, which will be analysed in section\todo{Link this}.

Tasks have a predetermined length (as saved in the trace), and they also hold some data that is used for scheduling such as their arrival time in the system, and how long they should wait until execution. 
Some properties such as required CPUs were removed.
My contribution of the phases-and-power model was added to the tasks.

The Scheduler policies are also marked green, as I added scheduler policies that could use this new model. Specifically, I added a (non-) checkpoint \& resume scheduler that works on oracle carbon prediction, this will be further explained in the latter sections.

Scheduled Tasks are submitted to a cluster.
In my case, this is a simulated, already implemented, cluster that simply logs when each task is entered and finished. 
In the original paper, they also used an open source adapter that would submit the tasks to an actual Slurm cluster. 
While this approach could have been used, I decided against it on the basis of not adding further complexity. 
Thus I also removed the Slurm adapter from my implementation.

% \paragraph{Data being used}
% 
% here i could describe which data is already being used (the traces, aswell as the historical carbon data)
% \begin{itemize}
%     \item Welche Traces gibt es, wodurch werden die characterisiert? (Länge, Anzahl, etc, etc) Vllt. kann man hier nen coolen vergleich erstellen, Auch könnte man ein paar Sätze darüber schreiben, wie die bisher in GAIA aufgenommen wurden.
%     \item Wie den scorelab trace benutzen und übersetzen? Gerne auf ner halben Seite aufschlüsseln, was die einzelnen Attribute aus sacct bedeuten.
%     \item Ansonsten kann man noch die dynamic ernergy sachen als Datenquelle auflisten, bzw. das mini experiment mit fmnist und roberta 
% \end{itemize}

\section{Implementing simulation-schedulers using the new model}

I would first like to summarize assumption that GAIA makes on the problem of carbon aware scheduling:

\begin{enumerate}
    \item job lengths are known
    \item all considered jobs are batch jobs and can thus be shifted temporally based on user deadlines
    \item carbon information is provided for near future time spans, no error is considered
    \item there are no hardware limitations or considerations, all jobs are executed in an isolated manner
    \item jobs have a constant power draw and can be checkpointed \& resumed for free
\end{enumerate}

Especially the last point will be modified going further. As shown in section \ref{sec:improving_the_model}, jobs now have a predefined, variable power draw according to the defined phases. Resuming a job also carries an additional startup phase with it.

\subsection{Modifying the existing jobs to use the new model}

The first step to be taken was to add the described model to the jobs as outlined in figure \ref{fig:class_diagram}. 
So far, jobs were generated from \verb|csv|-formatted traces, one example being provided in listing \ref{list:trace_csv}.

\begin{lstlisting}[frame=single, numbers=none, caption={Excerpt from the Alibaba-PAI trace}, label={list:trace_csv}, basicstyle=\ttfamily]
arrival_time,   length,     cpus
0.0,            6302.0,     1
68.0,           1000.0,     1
463.0,          1570.0,     3
838.0,          23549.0,    1
\end{lstlisting}

In the original GAIA implementation, jobs have a \verb|arrival_time|, which signify when the job should be added to the simulation in seconds. 
The \verb|length| of a job similarly encodes the seconds a job needs to fully execute. The \verb|cpu| column was not interesting for my work, as previously discussed.

The chosen way of supplying the model information was to add two new columns to the trace files, which would entail the type of job and a column for arguments. 
As of now, the following types are supported: \verb|constant|, \verb|mocked-constant-from-phases|, \verb|phases|, and \verb|ml|. 
The first two \verb|constant| types are for backward-compatibility with GAIA, and would be used in the evaluation later. 

Type \verb|phases| is the main way of creating a job with phases such as the one used in the experiment. The supplied argument in that case would entail a python readable \verb|dict|, essentially a \verb|JSON|-like definition of the model as described in listing \ref{listing:model_python}.
As this was just a string, I would then use Python's \verb|eval()| to read it back into a python object.\todo{Check how and if I should visualize this}
As an example, a simplified \verb|roberta.py|, the job that was measured in section \ref{sec:power_measurements}, could look like the code in listing \ref{list:roberta_model_definition}

\begin{lstlisting}[frame=single, numbers=none, caption={Simplified definition for a job similar to the experiment}, label={list:roberta_model_definition}, basicstyle=\ttfamily]
{
    'startup': [
        {'name': 'Start', 'duration': 5.34, 'power': 59.9},
        {'name': 'Finish Imports', 'duration': 12.36, 'power': 53.77},
        {'name': 'Data loaded', 'duration': 5.75, 'power': 63.17}, 
    ],
    'work': [
        {'name': 'Train', 'duration': 8.17, 'power': 221.93}, 
        {'name': 'Evaluate', 'duration': 1.54, 'power': 134.0}, 
        {'name': 'Save', 'duration': 2.72, 'power': 105.1,
            'is_checkpoint': True}, 
    ] * 5
}
\end{lstlisting}

In order to use this model definition and read out the power of a job at a given time, a function was crated that essentially traverses all phases in order, keeping track of the time, and returning the power of the phase when the requested time is reached. 
This was also used to create figure \ref{fig:model_overlaid}.

A generalizing helper type is \verb|ml|, which takes a dictionary of parameters and converts that to model parameters for the \verb|phases| type. A similar job as to the one above would be created with the following:

\begin{lstlisting}[frame=single, numbers=none, caption={Generic model definition for machine learning jobs}, label={list:roberta_model_definition_generic}, basicstyle=\ttfamily]
{
    'start_duration': 23.45
    'start_power': 60
    'training_duration': 8.17
    'training_power': 221.93
    'evaluate_duration': 1.54
    'evaluate_power': 63.17
    'save_duration': 2.72
    'save_power': 105.1
    'epochs': 5
}
\end{lstlisting}

\subsection{Uninterrupted oracle scheduling} \label{sec:uninterrupted_oracle_scheduling}

There already was an existing version of this scheduling approach which assumed constant power and that the carbon trace would be fully known ("oracle"). 
Jobs would not be able to checkpoint \& resume in this version.
The algorithm for which is pseudocoded in listing \ref{list:pseudocode_oracle}.

\begin{lstlisting}[frame=single, numbers=left, caption={Pseudocode for the original non-interrupt oracle scheduler}, label={list:pseudocode_oracle}, basicstyle=\ttfamily]
function schedule_job_no_checkpointing_resuming(job):
    possible_starts = []
    for every possible starttime until deadline:
        calculate carbon_emissions_of_job at starttime
        add carbon_emission to possible_starts
    
    return starttime with least carbon emissions

function carbon_emissions_of_job(job, carbon_trace):
    sum_of_emissions = 0

    for each second of job:
        sum_of_emissions += carbon_trace at timepoint * constant_watt
    
    return sum_of_emissions
\end{lstlisting}

In order to adjust this to the new model where power is a function of time, only line 13 needed to be changed. 
Thus, I implemented a python function whose input is the model parameters (see listing \ref{roberta_model_definition}), and which would return the power at a specified time. 
Evaluating this scheduling approach will take place in section \ref{sec:evaluate_scheduling}, after another scheduling algorithm is introduced.


\subsection{{Checkpoint \& resume scheduling for heterogeneous jobs}} \label{sec:checkpoint_resume_lp}

Under the constant-power, no startup-overhead assumption of the already existing implementation, the algorithm for this scheduler worked like the following:

\begin{itemize}
    \item order all time slots until the job deadline by their emissions ascending
    \item execute the job on those time slots until done
\end{itemize}

This would minimize the carbon emissions, as only the best time slots would be used, but would also create unrealistic schedules sometimes. 
For example, in figure \ref{fig:schedule_problems}, if a job's deadline is in the distant future, due to the diurnal nature of the electric grid, the job would be scheduled on the "peaks" of each day. Sometimes this would lead to a job being simulated as being executed for very short timeframes at a time\todo{Check wording}. As shown in the power measurement chapter, this kind of scheduling would create a lot of overhead from starting so many times.

\begin{figure}
    \missingfigure{I should show a Gantt chart of the mishappen scheduling for the checkpoint resume scheduler}
    \caption{Problems in the checkpoint \& resume scheduling under the previous assumptions}
    \label{fig:schedule_problems}
\end{figure}


In order to implement a scheduler for the improved job model, I chose to use \emph{linear programming} (also: LP).
In a nutshell, LP is an optimization method, that given linear equations and constraints, finds a set of variables (or a vector) that would minimize (or maximize) an optimization goal. 
The reason for choosing this approach over, for example coming up with a "standard" algorithm, was that I had never used LP before and that due to the "inherent optimization", the burden of\todo{Kann man das so sagen?} proof for the correctness would be lower.

Translating this to the context of carbon-aware scheduling, the resulting vector would decode when a job should be executed, the optimization goal would be to minimize the emitted carbon. 
A constraint for example could be that all time slots being executed should add up to the job length. 
The big challenge is then to model all of this in the form of mathematical equations, which will be most part of this section's remainder.

\paragraph{Personal experience with linear programming}

Linear programming is its own programming paradigm. Unlike, for example, imperative programming, where each instruction would be coded explicitly, enabling classic debugging and such, in LP, a mathematical model would be written which is then solved by a \emph{solver}. 
Different implementations for these solvers exist, but atleast in my case, the result output by a solver would either be an error, or the result set. 
As the solver uses the whole model simultaneously, "debugging" individual parts of the model is not possible outside removing them from the overall model.

Thus, an iterative approach to LP proved useful. 
I would encode some part of my overall problem into the model, solve it, and immediately plot the variables of the result set and check them visually.
The visual check is very important. 
Many times the solver would solve the problem "so well" that jobs would get scheduled in a way that the emitted carbon is minimized, but the jobs would be executed in edge cases that I did not intend, but that are possible under the model, examples of this will be given in the following parts.

Another part of programming LP is "just knowing the tricks". 
Many common programming operations, especially ones for control flow, cannot be expressed directly (as they need to be a linear equation).
There are some mappings for such operations, but finding out how seems to be harder than in other paradigms, since the LP community appears to be smaller in comparison and online resources are limited.
Some "tricks" will be highlighted in the latter parts as well. 

\paragraph{Modelling carbon-aware scheduling in a linear program}

In this part, I would like to preset the steps taken to model the scheduler as an LP problem, highlight some implementation details and visualizing the results of each step.
I used python's \verb|PuLP| library.
It allows creating LP models to standard file types and also helps in calling external solvers as well as querying the results.

\subparagraph{Executing the job}

Firstly, as the job has a certain length and a set deadline, I chose to use an array of time slot-variables as my result set. Each such variable would be a boolean signifying whether the job would be executed at that time.
For the first constraint, the sum of all boolean variables (which are represented as 0 and 1) should add up to the job length, effectively executing the job for length-many time slots.
The necessary objective function is to be defined as each time slot-being-executed multiplied by the carbon-per-watt at that time slot.

As an example, the code for this is shown in listing \ref{fig:lp_work}.

\begin{lstlisting}[frame=single, numbers=left, caption={LP Implementation for basic scheduling}, label={list:lp_work}, basicstyle=\ttfamily]
prob = LpProblem("CarbonAwareScheduling", pulp.LpMinimize)
work = LpVariable.dicts("work", (t for t in range(DEADLINE)), cat="Binary")

prob += lpSum([work[t] * carbon_cost_at_time[t] for t in range(DEADLINE)]) 
prob += lpSum(work[t] for t in range(DEADLINE)) == WORK_LENGTH

solver = pulp.Gurobi_CMD(timeLimit=timelimit)
prob.solve(solver)
\end{lstlisting}

The first line creates an LP problem and defines the optimization goal to be a minimizing one. 
Then, a variable "work" is defined as being \verb|DEADLINE|-many (in this case 300) integer-indexed boolean variables. 
These will be modeled as the time slots.
Line 4 defines the optimization goal; each time slot will be multiplied by the carbon emitted there. As one factor is a boolean, this will only increase the sum if the time slot is scheduled. 
The last two lines start the solving library. 
\verb|PuLP| offers an open source solver by default, but during development this proved to be very slow and also did not offer features such as time limits and retrieving intermediate results.
I thus chose the \emph{Gurobi} solver, a commercial solver that also offers academic licenses. 
That one supported the previously lamented features and increased development time by a lot. 

The running example will be a deadline of 300 time slots, the job has a startup phase of 20 units and has 100 work units. 
The above code leads to a result set that is visualized in figure \ref{fig:lp_work},
which on the bottom part shows the carbon emissions at each time slot and the upper graph shows the chosen time slots.
Looking at the figure, the results are not too surprising: the chosen time slots are just the points where emissions are lowest, essentially being equal to the result of the WaitAWhile-Algorithm.

\begin{figure}
    \includegraphics[width=\linewidth]{GAIA/notebooks/lp_work.pdf}
    \caption{Visualizing the result set of executing the job for its length}
    \label{fig:lp_work}
\end{figure}

\subparagraph{Adding overhead via startup phases}

In order to add a cost to starting a job, the code following in listing \ref{list:lp_overhead} is complemented, some parts are excluded for brevity.

\begin{lstlisting}[frame=single, numbers=left, caption={LP Implementation for overhead}, label={list:lp_overhead}, basicstyle=\ttfamily]
for t in range(DEADLINE - 1):
    prob += startup_finished[t] >= work[t + 1] - work[t]
    prob += startup_finished[t] + work[t] <= 1
    prob += starting[t] + work[t] <= 1

for i in range(STARTUP_LENGTH - 1, DEADLINE):
    prob += pulp.lpSum([starting[i - j] for j in range(STARTUP_LENGTH)]) 
        >= STARTUP_LENGTH * startup_finished[i]
\end{lstlisting}

In this snippet, two extra dictionary-variables are introduced, \verb|startup_finished| and \verb|startup|.
For every time slot (line 1), set \verb|startup_finished| to true if and only if there is a 0 to 1 transition along the \verb|work| dictionary. This becomes apparent when looking at the following truth table \ref{tab:truth_table_startup_finished}.
Notice how the numerical booleans help in this case, as negative integers get mapped to \verb|0|, or \verb|false| respectively.

\begin{table}[h!]
\centering
\begin{tabular}{|c|c|c|c|}
\hline
    $work[t]$ & $work[t+1]$ & $work[t+1]$ - $work[t]$ & $startup\_finished[t]$ \\ \hline
    $0$ & $0$ & $0$ & $0$ \\ \hline
    $0$ & $1$ & $1$ & $1$ \\ \hline
    $1$ & $0$ & $max(-1, 0) = 0$ & $0$ \\ \hline
    $1$ & $1$ & $0$ & $0$ \\ \hline
\end{tabular}
\caption{Truth table for finding when working time slots begin}
\label{tab:truth_table_startup_finished}
\end{table}

Line 3 and 4 ensure that "working" time slots and "startup" time slots are mutually exclusive.
The last loop defined that if (\verb|* startup_finsihed[i]|) a startup must be finished by some time slot, the previous \verb|STARTUP_LENGTH|-many time slots must be used for starting the job.

At this point, the solver scheduled jobs in a way that would sometimes execute the job at the very first time slots, as no startup could happen before time slot $0$, thus minimizing carbon but obviously producing a bogus result.
I thus added an extra constraint that work can only be scheduled after atleast the length of the startup phase.
The emitted carbon goal is changed to also include these new startup time slot, similarly to the previous listing \ref{list:lp_overhead}.
With this step, the result already looks more like a reasonable schedule, as shown in figure \ref{fig:lp_overhead}. 
This time, the optimal schedule includes executing the job in one go.
Doing a visual check, the job is also executed on seemingly the lowest carbon intensities.

\begin{figure}
    \includegraphics[width=\linewidth]{GAIA/notebooks/lp_overhead.pdf}
    \caption{LP Scheduling with overhead in form of a startup phase}
    \label{fig:lp_overhead}
\end{figure}

\subparagraph{Adding a notion of progress}

So far, the assumption of constant power has been continued.
Changing this assumption under the previous other scheduler using the greedy algorithm in section \ref{sec:uninterrupted_oracle_scheduling}, was relatively easy, as changing the constant expression to python function of the model sufficed. 
This cannot be reused in LP however, as the problem definition may only include linear equations, which a generic function is not. 
In order to add dynamic power into the linear program, I decided to \emph{linearize} the function, meaning the model needed to be split up into multiple linear approximations.

As the model already is essentially a step function of time to power, each phase being one step, the mapping is inherently pretty close. 
As such, \emph{a lot} of equations were needed that each express "if the progress in the job is $t$, set power to $model(t)$". 
For that a notion of progress and time is needed inside the model.
The progress inside each startup-phase needs to reset, as that can happen multiple times during the schedule. 
On the other hand, the progress for the productive work must not be reset between execution blocks. 

I added the following code to express this:

\begin{lstlisting}[frame=single, numbers=left, caption={Progress Variables in LP}, label={list:lp_progress}, basicstyle=\ttfamily, breaklines]
# define "work_time_progressed" and "startup_time_progressed" as 
# DEADLINE-many integer variables

M = DEADLINE * 2
for t in range(DEADLINE-1):
    if (t>0):
        prob += startup_time_progressed[t] >= startup_time_progressed[t-1] + 1 - (1 - starting[t]) * M 
        prob += startup_time_progressed[t] <= startup_time_progressed[t-1] + 1 + (1 - starting[t]) * M
    prob += startup_time_progressed[t] <= starting[t] * M 
    prob += work_time_progressed[0] == 0
    if (t > 0):
        prob += work_time_progressed[t] == work_time_progressed[t-1] + work[t]
\end{lstlisting}

On a base level, the idea is to count up each progress variables by 1, if a time slot is being determined for work or startup respectively (see line 12 for this).

This snippet also includes an LP "trick", namely the \emph{Big M Method}. 
In line 4, a constant \verb|M| is defined as "a \emph{large} integer, that cannot otherwise occur in the result set", in this case "large" being twice the amount of time slots but it could also be some other arbitrarily-chosen large number.

Take line 9 as an example: 
remember that \verb|starting[t]| is either $0$ or $1$, multiplying this by a large number means the right side is either $0$ or "\emph{large}". 
Constraining a variable to be less that \verb|M| effectively does nothing (is "\emph{relaxed}"), as every value of that variable is less than \verb|M| by definition of \verb|M|. 
The whole can expression be translated to "if a time slot is not used for starting the job, set the progress to 0, otherwise ignore this constraint", the Big M Method enables a way to add conditional constraints to a model!

While line 9 this defines the \verb|startup_progress| outside the startup phases, line 7 and 8 are needed to increase the progress by exactly 1. 
Notice how depending on the type of inequality, \emph{M} is either added or subtracted to the equation, a conditional "greater than"-relation is relaxed by setting the constraint to 0.

All in all, there is now a way to keep track of a job's progress inside the model, as shown in figure \ref{fig:lp_progress}'s upper graph. Attention should be drawn to the \verb|work_progress| which keeps its value even when time slots are not used for work. 
The schedule found by the solver is not different to the previous one which added overhead, as the optimization goal was not changed and the progress indicators have no impact on the scheduling (yet).

\begin{figure}
    \includegraphics[width=\linewidth]{GAIA/notebooks/lp_progress.pdf}
    \caption{The result set now contains the progresses of the startup and work phase}
    \label{fig:lp_progress}
\end{figure}

\subparagraph{Adding dynamic power according to the new model}

The goal of this part will be to have an LP variable for each phase, which indicates when the job is in that phase.
When such indicators exist, the formula for calculating a schedule's carbon emissions can be changed to the following (\ref{formula:total_carbon}):

\begin{align}
    \label{formula:total_carbon}
    carbonEmitted(t) = \sum_{p \in Phases} isActive(p, t) * powerOfPhase(p) * carbonEmission(t)
\end{align}

With that goal in mind, we add the pseudocode listed in \ref{list:lp_phases} to determine when a phase is active. 

\begin{lstlisting}[frame=single, numbers=left, caption={Phase detection in LP}, label={list:lp_phases}, basicstyle=\ttfamily, breaklines]
# for each phase
# let phase_indicator, phase_indicator_upper, phase_indicator_lower be DEADLINE-many boolean variables

# set lower_bound to be the minimum progress this phase can occur in 
# set upper_bound to be the maximum similarly

# do the following for each phase
for t in range(DEADLINE):
    prob += progress[t] - lower_bound <= M*phase_indicator_lower[t]
    prob += lower_bound - progress[t] <= M*(1-phase_indicator_lower[t])

    prob += upper_bound - progress[t] <= M*phase_indicator_upper[t]
    prob += progress[t] - upper_bound <= M*(1-phase_indicator_upper[t])

    prob += phase_variable[t] >= phase_indicator_lower[t] + phase_indicator_upper[t] - 1
    prob += phase_variable[t] <= phase_indicator_lower[t]
    prob += phase_variable[t] <= phase_indicator_upper[t]
\end{lstlisting}

Unwrapping this, two helper variables are added per phase per time slot. 
A lower variable indicates that the previously established progress is above the threshold for a phase, while the upper variable indicates the opposite. 
The \verb|phase_variabel| is then "active" where these two overlap (line 14 to 17 define a logical "AND").

The constraint of line 8 is only applied if the indicator is \verb|false|.
If it is false, the progress at the time slot must be below the lower bound (as negative numbers are equal to 0).
Line 9 works on the negated indicator, applying the constraint if it is \verb|true|. 
If the indicator is \verb|true|, progress must be higher than the lower bound. 
Combining this, the expression \verb|phase_indicator_lower[t] <=> progress[t] > lower_bound| is added to the model. 
The upper bound is formulated in the following two lanes similarly.

Using this addition, the schedule now looks like the one shown in figure \ref{fig:lp_states}. 
Unlike the previous times, splitting the job up into two parts is now the optimal solution. 
Attention should be drawn to the circumstance that the high-powered phase (the green one) is scheduled on the lowest carbon emissions while the low-powered one is scheduled on the higher emission time slots.

\begin{figure}
    \includegraphics[width=\linewidth]{GAIA/notebooks/lp_states.pdf}
    \caption{The final scheduling including dynamic power via phases}
    \label{fig:lp_states}
    \todo[inline]{These graphs needs some more love, the legend is pretty bad.}
\end{figure}

\subparagraph{Issues of the implementation}

As of now, checkpointing may happen at any point in the scheduler, as the \verb|work_progress| is never reset. 
Checkpointing at specific points, such as after specific phases like in the entry ML example in section, will be left for future work.\todo{Add this to future work!}

Another issue is the runtime and hardware requirements for finding a solution.
While the upper example of 250 time slots and a job length of 120 were found within minutes on my laptop (a Lenovo T470 with 8GB RAM and an i5-7200U CPU @ 2.50GHz), bigger problem sizes such as 5000 time slots and a length of 800 would barely finish within 30 minutes with a \emph{gap}-value of 98\%.
As the LP solver computes the problem, the gap value is an indicator of how close the solver is to finding the optimal solution, the optimal solution is found when the gap is at 0\%.

Solvers are able to take advantage of multiple CPU cores, but in my case, the issue was probably a lack of memory. 
Glancing into the system monitor utility on Ubuntu would show that all 8 GB are in use and that swap space is being used a lot.

\subparagraph{Decreasing the Problem Size}

In the previously shown Figures, each carbon-emission data point would correspond to one time slot, which in turn corresponds to one unit of time in the job. 
Going by the resolution of the carbon-emissions, each data point describes the emissions for one hour. 
In the original \verb|GAIA| implementation, job lengths are given in seconds, however. 
The example workload used for measuring in Section \ref{sec:power_measurements} also showed that timescales in seconds are useful. 

Using a 1:1 mapping between those would result in 3600 (60s * 60) time slots for each hour, resulting in very large problem sizes in turn needing long execution times and high amounts of memory.

In order to improve on the issues mentioned above, the following optimization is made: by calculating the \emph{greatest common divisor (gcd)} between all phase durations and the time resolution of the carbon-emissions (3600), all durations can be scaled down by the \verb|gcd|. Each time slot in the model would then represent \verb|gcd|-many seconds.

The effect of this optimization heavily depends on the input phases. 
If there are very short phases, the \verb|gcd| would likewise be low and problem size would not be decreased much. 
